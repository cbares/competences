\documentclass{article}

\usepackage[
% nomargin, % Remove margin competence tag
]{competences}

\renewcommand \thesection{\Alph{section}}

\begin{document}

\declareprefix{CA}
\declareprefix{CB}
\declareprefix{CC}

% Declare competences is not mandantory,
% but it defines order in sumup tables
% and allows to detect forgiven competences.
\declarecompetence{CA1}
\declarecompetence{CA2}
\declarecompetence{CB1}
\declarecompetence{CB2}
\declarecompetence{CC1}
\declarecompetence{CC2}
\declarecompetence{CC3}
\declarecompetence{CC4}
\declarecompetence{FOO}

% Some competences are global,
% ie not link to a particular question.
\addGlobalCompetence[1]{CC4}
\addGlobalCompetence[2]{CC3}


\section{Test}
Question one.
\addcompetence{CA1}

\section{test2}
Question two with 2 competences.
\addcompetence{CA1}\addcompetence{CA2}

\subsection{test3}
Question three with points different from 1.
\addcompetence[3]{CB1}\addcompetence[1]{CC1}

\subsubsection{test4}
Question four in a subsubsection.
\addcompetence[3]{CB1}


\section{test5}
\subsection{test6}
\begin{enumerate}
 \item Question five in enumerate \addcompetence[1]{CB2}
 \item Question six \addcompetence[1]{CC2}
  \begin{enumerate}
    \item text seven in enumerate in enumerate\addcompetence[3]{CC1}
    \item text hight \addcompetence[1]{CB2}
  \end{enumerate}
\end{enumerate}

\section{Test 7}
\subsection{test10}
  \begin{enumerate}
    \item text nine in enumerate\addcompetence{CB1}
 \end{enumerate}


\section{Résumé des compétences}
\tableaupartie{A}
% \tableaupartie{A}
\tableaupartie{B}
\tableaupartie{C}
%\tableaupartie{D}

\tableaupartie{Global}


\tableauprefix{A,B,C,Global}

\end{document}
